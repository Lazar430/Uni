\documentclass[12pt, letterpaper]{article}

\usepackage{amsfonts}
\usepackage{amsmath}

\usepackage[romanian]{babel}
\usepackage{combelow}

\setlength\parindent{0pt}

\title{Note Curs Analiz\u{a} Matematic\u{a}: \cb{S}iruri \cb{s}i Serii Numerice}
\author{LazR}
\date{22 Octombrie 2023}

\begin{document}

\maketitle

\newpage

\section*{\cb{S}iruri de numere reale}

\textbf{Definitie:} Un \cb{s}ir de numere reale este o func\cb{t}ie $x : \mathbb{N} \rightarrow \mathbb{R}$ care
asociaz\u{a} fiec\u{a}rui num\u{a}r natural $n$ num\u{a}rul naturan $x_n$. Nota\cb{t}ia tipic\u{a} este
$(x_n)_{n \in \mathbb{N}}$ sau $(x_n)_{n \geq 0}$.\\

\textbf{Definitie:} Un \cb{s}ir de numere reale $(x_n)$ se nume\cb{s}te \cb{s}ir m\u{a}rginit dac\u{a}
exist\u{a} un num\u{a}r real pozitiv $M > 0$ astfel \^{i}nc\^{a}t

\[ |x| \leq M \hspace{3pt} \forall n \in \mathbb{N}  \]

\end{document}
